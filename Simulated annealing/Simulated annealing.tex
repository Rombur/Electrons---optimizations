\documentclass{article}
\usepackage{amsmath}
\usepackage{array}
\usepackage{color}
\usepackage{graphicx}
\usepackage{float} %utiliser H pour forcer � mettre l'image o� on veut
\usepackage{lscape} %utilisation du mode paysage
\usepackage{mathbbol} % permet d'avoir le vrai symbol pour les reels grace a mathbb
\usepackage{enumerate}
\usepackage{marvosym}	
\usepackage{moreverb} % permet d'utiliser verbatimtab : conservation la tabulation 


\setlength {\textwidth}{16cm}
\setlength {\textheight}{21cm} 
\setlength {\oddsidemargin}{0cm}
\setlength{\headsep}{5pt} 

\newcommand\bn{\boldsymbol{\nabla}}
\newcommand\bo{\boldsymbol{\Omega}}
\newcommand\br{\mathbf{r}}
\newcommand\la{\left\langle}
\newcommand\ra{\right\rangle}
\newcommand\bs{\boldsymbol}
\newcommand\red{\textcolor{red}}

\renewcommand{\(}{\left(}
\renewcommand{\)}{\right)}
\renewcommand{\[}{\left[}
\renewcommand{\]}{\right]}

\newtheorem{theorem}{Theorem}[section]

\begin{document}
\title{\textsc{\huge{Simulated annealing }}}
\author{Bruno Turcksin} 
\date{}
\maketitle

\section{Simulated annealing}
We try to solve the well-known problem using the simulated annealing. The advantage of the simulated annealing over the deterministic method is that it can escape a local minima. Unlike the deterministic methods which have to ``go down'' the objective function, the simulated annealing can ``go up''. The disadvantage of the simulated annealing is that the convergence can be very slow.

We try 

\section{Equations}
We recall briefly the equations :
\begin{align}
f &= \sum_{(i,j)\in \mathcal{T}}(D_{ij}-\delta_{ij})^2\\
h_1 &= \gamma - \sum_{(i,j)\in \mathcal{D}} \frac{V_{ij}}{V} \nu\(D_{ij}-\delta_{ij}\)\geq 0\\
h_2 &= J \geq 0
\end{align}

\end{document}
