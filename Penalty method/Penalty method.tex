\documentclass{article}
\usepackage{amsmath}
\usepackage{array}
\usepackage{color}
\usepackage{graphicx}
\usepackage{float} %utiliser H pour forcer � mettre l'image o� on veut
\usepackage{lscape} %utilisation du mode paysage
\usepackage{mathbbol} % permet d'avoir le vrai symbol pour les reels grace a mathbb
\usepackage{enumerate}
\usepackage{marvosym}	
\usepackage{moreverb} % permet d'utiliser verbatimtab : conservation la tabulation 


\setlength {\textwidth}{16cm}
\setlength {\textheight}{21cm} 
\setlength {\oddsidemargin}{0cm}
\setlength{\headsep}{5pt} 

\newcommand\bn{\boldsymbol{\nabla}}
\newcommand\bo{\boldsymbol{\Omega}}
\newcommand\br{\mathbf{r}}
\newcommand\la{\left\langle}
\newcommand\ra{\right\rangle}
\newcommand\bs{\boldsymbol}
\newcommand\red{\textcolor{red}}

\renewcommand{\(}{\left(}
\renewcommand{\)}{\right)}
\renewcommand{\[}{\left[}
\renewcommand{\]}{\right]}

\newtheorem{theorem}{Theorem}[section]

\begin{document}
\title{\textsc{\huge{Penalty method}}}
\author{Bruno Turcksin} 
\date{}
\maketitle
\tableofcontents

\section{Penalty method}
Instead of minimizing a constrained optimization problem, we solve an unconstrained problem :
\begin{equation}
\min Q_{\mu} (x) = f(x) + \frac{1}{2\mu} \|g(x)\|^2 + \frac{1}{2\mu} \|\[h(x)\]^-\|^2
\end{equation}
where :
\begin{equation}
\[h(x)\]^- = \min \{0,h(x)\}
\end{equation}
We decrease the value of the parameter $\mu$ during the optimization process.

\section{Equations}
We have the following objective function :
\begin{equation}
f = \sum_{(i,j)\in \mathcal T} \(D_{ij} - \delta_{ij}\)^2
\end{equation}
The constraints are given by :
\begin{align}
&h_1 = \gamma - \sum_{ij \in \mathcal{D}} \frac{V_{ij}}{V} \nu \(D_{ij} - \delta_{ij}\) \geq 0 \label{dv}\\
&h_2 = J \geq 0
\end{align}
Thus, we have the following quadratic relaxation :
\begin{equation}
Q_{\mu} = \sum_{(i,j)\in \mathcal T} \(D_{ij} - \delta_{ij}\)^2 + \frac{1}{2\mu} \(\[\gamma - \sum_{(i,j) \in \mathcal{D}} \frac{V_{ij}}{V} \nu \(D_{ij} - \delta_{ij}\)\]^-\)^2 + \frac{1}{2\mu} \(\[J\]^-\)^2
\end{equation}
If we want to use the steepest descent or a Newton's method to solve the previous problem, we need to take the derivative of the Heaviside function. That is a problem and thus we use the following \hbox{approximation :}
\begin{equation}
\nu (x) = \frac{1}{1+e^{-2kx}}
\end{equation}
where $k$ is a large constant.
%bjfew
The gradient of the quadratic relaxation is :
\begin{equation}
\begin{split}
\bn Q_{\mu} &= \sum_{(i,j) \in \mathcal{T}} 2 \(D_{ij} - \delta_{ij}\) \bn D_{ij} + \frac{1}{\mu} \(\[\gamma - \sum_{(i,j) \in \mathcal{D}}\frac{V_{ij}}{V} \frac{1}{1+e^{-2k\(D_{ij} - \delta_{ij}\)}}\]^-\) \times\\
&\(\sum_{(i,j)\in \mathcal{D}}\frac{V_{ij}}{V} \frac{-2k e^{-2k\(D_{ij}-\delta_{ij}\)} \bn D_{ij}}{\(1+e^{-2k(D_{ij}-\delta_ij)}\)^2}\)+ \frac{1}{\mu} \(\[J\]^-\)
\end{split}
\end{equation}
We see here that there is a problem with the penalty method. If we look at the second term of the previous equation, we see that the penalty term disappear when the constraint is satisfied (like it should) but also that the penalty term can be arbitrary close to 0 by increasing the dose. The problem is that the derivative of the logistic is very close to 0 and get closer to 0 when $x$ increases. Thus, by increasing slightly the dose, we can decrease that term faster than we decrease $\mu$.

\section{Modified constraints}
We see that we need to modified the constraints if we want to use the penalty method. To do that, we replaced the Heaviside function $\nu(x)$ by the following function :
\begin{equation}
\tilde{\nu}(x) = \left\{
\begin{aligned}
&x & \textrm{if } x\geq 0\\
&0 & \textrm{else}
\end{aligned}
\right.
\end{equation}
Thus, (\ref{dv}) becomes :
\begin{align}
\widetilde{h}_1 = \gamma - \sum_{ij\in\mathcal{D}} \frac{V_{ij}}{V} \frac{\tilde{\nu}\(D_{ij}-\delta_{ij}\)}{\delta_{ij}} \geq 0
\end{align}
where we had to renormalize the new function to keep $\gamma$ adimensional. So, we have the new quadratic relaxation :
\begin{equation}
Q_{\mu} = \sum_{(i,j)\in \mathcal{T}} (D_{ij}-\delta_{ij})^2 + \frac{1}{2\mu} \(\[\gamma - \sum_{(i,j) \in \mathcal{D}} \frac{V_{ij}}{V} \frac{\tilde{\nu} \(D_{ij} - \delta_{ij}\)}{\delta_{ij}}\]^-\)^2 + \frac{1}{2\mu} \(\[J\]^-\)^2
\end{equation}
The gradient of the constraint is given by :
\begin{equation}
\begin{split}
\bn Q_{\mu} &= \sum_{(i,j) \in \mathcal{T}} 2 \(D_{ij} - \delta_{ij}\) \bn D_{ij} + \frac{1}{\mu} \(\[\gamma-\sum_{(i,j)\in\mathcal{D}} \frac{V_{ij}}{V} \frac{\tilde{\nu}(D_{ij}-\delta_{ij})}{\delta_{ij}}\]^-\) \times\\
&\sum_{(i,j)\in \mathcal{D}} \frac{V_{ij}}{V} \frac{\[- \bn D_{ij}\]^-}{\delta_{ij}}  + \frac{1}{\mu}\(\[J\]^-\)
\end{split} 
\end{equation}
The Hessian is given by : 
\begin{equation}
\begin{split}
\bn^2 Q_{\mu} &= \sum_{(i,j)\in \mathcal{T}} 2 \(\bn D_{ij}\) \cdot \(\bn D_{ij}\)^T + \frac{1}{\mu} \(\sum_{(i,j)\in \mathcal{D}} \frac{V_{ij}}{V} \frac{\[-\bn D_{ij}\]^-}{\delta_{ij}}\) \cdot \\
&\(\sum_{(i,j)\in \mathcal{D}} \frac{V_{ij}}{V} \frac{\[- \bn D_{ij}\]^-}{\delta_{ij}} \)^T + \frac{1}{\mu} \[I\]^-
\end{split}
\end{equation}

\section{Results}
We will use the exact same tests that the ones we used for the project.
\subsection{Test 1}
We choose a convergence tolerance of $10^{-8}$ and $c=0.9$ ($\mu_0 = 100$, $tol_0$ = 100, $c_1 = 10^{-4}$ and $c_2 = 0.9$). The initial value of $\mu$ was 100. It took 34.67s to solve the problem. The best value of the objective function is 0.84370144. The 2 beams have the same intensity : 0.8088. The doses for the 25 cells are given by :
\begin{table}[H]
\begin{center}
\begin{tabular}{|c|c|c|c|c|}
\hline
 0.00645411 & 0.00942717 & 0.01067242 & 0.00536458 & 0.00256368 \\ 
\hline
 0.02754867 & 0.0294389 & 0.02987271 & 0.01347729 & 0.00536458 \\
\hline
 0.20989893 & 0.11010246 & 0.08146778 & 0.02987275 & 0.01067244 \\
\hline
  0.03161122 & 0.04540035 & 0.11010165 & 0.02943874 & 0.00942713 \\
\hline
 0.01034451 & 0.03161104 & 0.20989697 & 0.02754845 & 0.00645408 \\
\hline
\end{tabular}
\end{center}
\end{table}
We can see that the dose in the target is 0.08146778 instead of 1. We have 4 cells which receive a dose larger than 0.1 (0.20989697,0.20989697,0.11010165,0.11010165). We see that the penalty method finds a symmetric solution even if the best solution is to turn off one of the 2 beams. If we use the following objective function :
\begin{equation}
\min \sum_{(i,j)\in \mathcal{T}} (D_{ij}-\delta_{ij})^2
\label{second}
\end{equation}
we find 0.843701439178128
\subsection{Test 2}
The best value of the objective function, using line search ($\mu_0 = 100$, $tol_0$ = 1000, $c_1 = 0.5\cdot 10^{-4}$ and $c_2 = 0.85$), was 1.56263043627. We choose a convergence tolerance of $10^{-2}$ and $c=0.8$. We changed the parameters so we can converge more easily. The intensity of the left beam was 0.4401704 and the intensity of the bottom beams was 1.23852802. The time needed was 186s. The dose was :
\begin{table}[H]
\begin{center}
\begin{tabular}{|c|c|c|c|c|}
\hline
 0.00477752 &  0.00787439 & 0.01195714 &   0.00566355  & 0.00266036 \\
\hline
 0.0175427 &  0.02267178 &  0.03258444  &   0.01398557 &  0.00547025 \\ 
\hline
 0.11861055 &  0.07307477 & 0.08454024 &  0.02941425 &  0.01019272  \\
\hline
 0.02376343 &  0.04711258 &  0.15543409 &  0.03842637 &  0.01169098 \\ 
\hline
 0.01073464 & 0.04184319  &  0.31701746  &  0.03963235  &  0.00861749 \\
\hline
\end{tabular}
\end{center}
\end{table}
The dose in the target is 0.08454024 and 0.15543409. There are 2 cells which receive a dose larger than 0.1 (0.11861055 and  0.31701746), and thus, the constraints are strictly respected. If we use (\ref{second}), we find 1.551358148513386. We should notice that is very difficult to satisfy the 2 Wolfe conditions for the step length and most of the time the step length is the smallest allowed ($10^{-2}$). The reason is that the first Wolfe condition (sufficient decrease) is not satisfied when the step length is large. However, when the step length decreases, the second Wolfe condition (upward curvature) is not satisfied anymore.

\subsection{20 beams}
Here we should notice that using the algorithm with the same parameters than usual did not work. At some point the algorithm just diverges without any reasons. The left and the right-hand-side were only small perturbations of the left and the right-hand-side of the previous step. The condition number was of the order of $10^10$ but the previous had the same condition number and there was not any problem. Changing the ``Levenberg-Marquardt'' from $10^{-8}$ to $10^{-5}$ fixed the problem. Here are the results that we found :\\
We used the following parameters : convergence tolerance of $10^{-2}$ and $c=0.95$, $\mu_0 = 100$, $tol_0$ = 1000, $c_1 = 0.5\cdot 10^{-4}$ and $c_2 = 0.85$. The best value was 0.769944569531 and the time needed was 17533s (however, there was a lot of things to write in a file, around $10^6$ lines). The values of the beams are :
\begin{table}[H]
\begin{center}
\begin{tabular}{|c|c|c|c|c|c|}
\hline
Side & \multicolumn{5}{c|}{Intensity}\\
\hline
Left & 1.85764121e-01 &  -1.26421726e-03  &   5.843062244e-01   &  -2.39997441e-04&  7.68831534e-01 \\
Right &  3.821645279e-01  &  1.87516795e-01  &  2.94335641e-01  & 2.48556839e-01  &  4.56235093e-01 \\
Bottom & 4.56235093e-01 & 2.485568446e-06  & 2.94335641e-01  &  1.87516795e-01 & 3.82645279e-01 \\
Top & 7.68834534e-01 & -2.39997441e-04 & 5.84307602e-01 & -1.26421726E-03 & 1.85762961E-01\\
\hline
\end{tabular}
\end{center}
\end{table}
For the dose, we have :
\begin{table}[H]
\begin{center}
\begin{tabular}{|c|c|c|c|c|}
\hline
0.14846951 & 0.22461436 & 0.22461436 & 0.10552206 & 0.18258683\\
\hline
0.09813131 & 0.13773937 & 0.09739758 & 0.12797325 & 0.39719311\\
\hline
0.22461403 & 0.13773924 & 0.12253715 & 0.10601778 & 0.14846948\\
\hline
0.10552206 & 0.09739758 & 0.1060178 & 0.10294864 & 0.13060944 \\
\hline
0.18258712 & 0.12797337 & 0.13472355 & 0.13060946 & 0.21044066 \\
\hline
\end{tabular}
\end{center}
\end{table}
We see that we have 21 cells above the limits (0.14846951, 0.22461436, 0.22461436, 0.10552206, 0.18258683, 0.13773937, 0.12797325, 0.39719311, 0.22461403, 0.13773924, 0.10601778, 0.14846948, 0.10552206, 0.1060178, 0.10294864, 0.13060944, 0.18258712, 0.12797337, 0.13472355, 0.13060946 and 0.21044066). The dose in the target is 0.12253715 instead of 1. If we use (\ref{second}), the value of the objective function is 0.769941053130123

\end{document}
